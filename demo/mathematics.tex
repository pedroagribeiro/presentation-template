\documentclass[12pt,a4paper]{article}

\begin{document}

greek letters:
$$\pi \approx 3.1415$$
$$\alpha + \beta$$
$$\varepsilon > 0$$

trigonometric functions:
$$\sin^2{x} + \cos^2{x} = 1$$
$$\tan{x} = \frac{\sin{x}}{\cos{x}}$$

log functions:
$$\log{x}$$
$$\log_2{x}$$
$$\ln{x}$$

square root:
$$\varphi=\frac{1 +\sqrt{5}}{2}$$
$$\sqrt[3]{8}=2$$

About $\displaystyle{\frac{2}{3}}$ of {\LaTeX} is fun. There is a problem
with brackets that you can solve with escape $\{a,b,c\}$. Other types of problems are the small brackets. $\displaystyle{3\left(\frac{2}{3}\right) = 2}$.

%https://www.johndcook.com/blog/2009/09/14/latex-multi-part-definitions/
$$
|x| =
\left\{
  \begin{array}{rr}
    %https://texblog.org/2012/09/24/normal-text-in-math-mode/
		x,  & \textrm{se } x \geq 0 \\
		-x, & x < 0
	\end{array}
\right.
$$

\begin{eqnarray*}
x^2+x-4&=&x \\
x^2 &=&4 \\
x &=& \pm 2 
\end{eqnarray*}

\end{document}
